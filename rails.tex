\subsection{O co chodzi?}
\begin{frame}[fragile]
\frametitle{Rails}
\begin{block}{Framework do szybkiego tworzenia aplikacji sieciowych}
\begin{itemize}
\item{Prosty serwer HTTP (Rack)}
\item{Integracja z wieloma bazami danych (SQL i NoSQL)}
\item{Przydatne rozszerzenia podstawowych klas Ruby'ego}
  \begin{itemize}
    \item{String, Time, \ldots}
  \end{itemize}
\end{itemize}
\end{block}
\end{frame}

\begin{frame}[fragile]
\frametitle{Rails}
\begin{block}{Filozofia Ruby on Rails}
\begin{itemize}
\item{Architektura MVC (Model - Widok - Kontroler)}
\item{Convention over Configuration}
\item{Test-Driven Development}
\end{itemize}
\end{block}
\end{frame}

\begin{frame}[fragile]
\frametitle{MVC}
\myfullimage{mvc.png}
\end{frame}

\mytitle{Gdzie wykorzystano Ruby on Rails?}

\subsection{Zastosowanie}
\begin{frame}[fragile]
\frametitle{GitHub}
\myfullimage{github.png}
\end{frame}

\begin{frame}[fragile]
\frametitle{Twitter (początkowo)}
\myfullimage{twitter.png}
\end{frame}

\begin{frame}[fragile]
\frametitle{Hulu}
\myfullimage{hulu.png}
\end{frame}
