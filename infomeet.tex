\documentclass{beamer}
\usetheme{Warsaw}
\usecolortheme{structure}
\usepackage{polski}
\usepackage[utf8]{inputenc}
\usepackage{hyperref}

\usepackage{listings}
\lstloadlanguages{Ruby}
\lstset{basicstyle=\ttfamily\color{black}\scriptsize,
        commentstyle = \ttfamily\color{red},
        keywordstyle=\ttfamily\color{blue},
        stringstyle=\color{orange},
        breaklines=true}

\hypersetup{colorlinks=true}

\title{Ruby on Rails}
\subtitle{z czym to się je?}
\author{Marek Waligórski}
\date{31 stycznia 2015}
\institute{InfoMEET Poznań}
\titlegraphic{\includegraphics[width=2cm,height=2cm]{Ruby_logo.png}}

% Dodaj obrazek do kazdego frametitle
\setbeamertemplate{frametitle}{%
    \nointerlineskip
    \begin{beamercolorbox}[sep=0.3cm,ht=1.8em,wd=\paperwidth]{frametitle}
        \vbox{}
        \vspace{-1.2ex}
        \strut\insertframetitle\strut
        \hfill
        \includegraphics[height=0.5cm]{Ruby_logo.png}
    \end{beamercolorbox}
}

\newcommand{\myurl}[1]{%
    \begin{block}{}
       \centering \huge
       \includegraphics[width=0.35cm]{link.png}
       \hspace{0.1cm}
       \url{#1}
    \end{block}
}

\begin{document}

\begin{frame}[fragile]
\titlepage
\end{frame}

\section{Ruby}
\begin{frame}[fragile]\frametitle{Ruby}
    \begin{itemize}
        \item 
        \item object oriented
        \item open classes, open modules, open everything
        \item mozna nadpisac singletonowe metody kazdego obiektu
    \end{itemize}
\end{frame}


\begin{frame}[fragile]
\frametitle{Wypróbuj Ruby'ego bez instalacji}
    \myurl{tryruby.org}
    \hspace*{-1.1cm}
    \includegraphics[width=\paperwidth]{tryruby.png}
\end{frame}
\begin{frame}[fragile]
\frametitle{Wypróbuj Ruby'ego bez instalacji}
    \myurl{www.bloc.io/ruby-warrior}
    \hspace*{-1.1cm}
    \includegraphics[width=\paperwidth]{rubywarrior.png}
\end{frame}

\begin{frame}[fragile]
\frametitle{Big sites made with RoR}
Hulu, YellowPages, GitHub, Twitter (początkowo), Ask.fm
\end{frame}

\begin{frame}[fragile]
\frametitle{Big sites made with RoR}
Hulu, YellowPages, GitHub, Twitter (początkowo), Ask.fm
\end{frame}

\begin{frame}[fragile]
\begin{lstlisting}[language=Ruby]

#this is a comment

a = 5
b = a * 5
puts b
puts b + 3

class Test < Test::SomeClass
@test = 3
    def bar
        if foo
            puts "foo"
        else
            puts "bar"
        end
    end
end
\end{lstlisting}
lalal
\end{frame}

\begin{frame}[fragile]
\frametitle{klasy}
\end{frame}
\begin{frame}[fragile]
\frametitle{Everything is an object}
\end{frame}
\begin{frame}[fragile]
\frametitle{bloki (closures)}
\end{frame}
\begin{frame}[fragile]
\frametitle{spaceship operator + sorting}
\end{frame}
\begin{frame}[fragile]
\frametitle{singleton classes}
\end{frame}
\begin{frame}[fragile]
\frametitle{CONSTANTS??}
\end{frame}
\begin{frame}[fragile]
\frametitle{Gems for everything}
 + Gemfile - so cool (bundle install)
\end{frame}
\begin{frame}[fragile]
\frametitle{Garbage Collection - Mark \& Sweep}
% http://patshaughnessy.net/2013/10/24/visualizing-garbage-collection-in-ruby-and-python
\end{frame}

\begin{frame}[fragile]
\frametitle{Wątki zielone}
\end{frame}
\subsection{Implementacje}
 - otwartość

\begin{frame}[fragile]
\frametitle{implementacje}
    * MRI
    - JRuby - różnice
\end{frame}



\subsection{Plusy}

\begin{frame}[fragile]
\frametitle{Co nam dał?}

- Technologie sieciowe - where it excels
* lessCSS / SASS
* HAML
* CoffeeScript
* Rails!!!
\end{frame}

\subsection{Z czym się tego nie je}

\begin{frame}[fragile]
\frametitle{Z czym się tego nie je?}
    * GameDev
    * High-performance tasks (ale można integrować z modułami C++, ale nie ma np.\ bytecode'u - tylko AST)
    * Wątki - sztuczne (JRuby obsługuje kilka)
\end{frame}

\section{Rails}
\subsection{O co chodzi?}
\begin{frame}[fragile]
\frametitle{MVC}

\end{frame}

\begin{frame}[fragile]
\frametitle{TDD?}
\end{frame}

\begin{frame}[fragile]
\frametitle{Jak zacząć?}

Heroku 
EngineYard
\end{frame}



\section{Podsumowanie}

\begin{frame}[fragile]
\frametitle{PRUG}
\end{frame}

\begin{frame}[fragile]
\frametitle{Big sites made with RoR}
Hulu, YellowPages, GitHub, Twitter (początkowo), Ask.fm
\end{frame}

\begin{frame}[fragile]
\frametitle{Rekrutujemy!}
\url{https://netguru.co/career}
\end{frame}
\end{document}