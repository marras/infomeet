\documentclass{beamer}
\usetheme{Warsaw}
\usecolortheme{structure}
\usepackage{polski}
\usepackage[utf8]{inputenc}
\usepackage{hyperref}
\usepackage{tikz}

\usepackage{listings}
\lstloadlanguages{Ruby}
\lstset{basicstyle=\ttfamily\color{black}\scriptsize,
        commentstyle = \ttfamily\color{red},
        keywordstyle=\ttfamily\color{blue},
        stringstyle=\color{orange},
        showstringspaces=false,
        breaklines=true}
\lstset{escapechar=£}

\usepackage{xcolor,tikz}
\newcommand\bh{\tikz[remember picture]
                \node (begin highlight) {};
             }
\newcommand\eh{\tikz[remember picture]
             \node (end highlight) {};
             \tikz[remember picture, overlay] 
             \draw[yellow,line width=10pt,opacity=0.3] (begin highlight) -- (end
              highlight);
             }

\hypersetup{colorlinks=true}

\title{Ruby on Rails}
\subtitle{z czym to się je?}
\author{Marek Waligórski}
\date{31 stycznia 2015}
\institute{InfoMEET Poznań}
\titlegraphic{\includegraphics[width=2cm,height=2cm]{Ruby_logo.png}}

% Dodaj obrazek do kazdego frametitle
\setbeamertemplate{frametitle}{%
    \nointerlineskip
    \begin{beamercolorbox}[sep=0.3cm,ht=1.8em,wd=\paperwidth]{frametitle}
        \vbox{}
        \vspace{-1.2ex}
        \strut\insertframetitle\strut
        \hfill
        \includegraphics[height=0.5cm]{Ruby_logo.png}
    \end{beamercolorbox}
}

\newcommand{\myfullimage}[1]{%
    \hspace*{-1.1cm}
    \includegraphics[width=\paperwidth]{#1}
}

\newcommand{\crossout}[0]{%
\begin{tikzpicture}[remember picture,overlay,decoration=penciline]
\draw[decorate,line width=15pt,red!60!black]
  (current page.north west) -- (current page.south east);
\draw[decorate,line width=15pt,red!60!black]
  (current page.north east) -- (current page.center) -- (current page.south west);
\end{tikzpicture}
}

\newcommand{\myurl}[1]{%
    \begin{block}{}
       \centering \huge
       \includegraphics[width=0.35cm]{link.png}
       \hspace{0.1cm}
       \url{#1}
    \end{block}
}
\newcommand{\mytitle}[1]{%
    \begin{frame}[plain]
        \begin{centering}
        \vspace{1em}\par
            \begin{beamercolorbox}[rounded=true,sep=4pt,center]{part title}
               \huge #1
            \end{beamercolorbox}
        \end{centering}
    \end{frame}
}
\newcommand{\mysection}[1]{%
    \section{#1}
    \mytitle{#1}
}

\begin{document}

\begin{frame}[plain]
\titlepage
\end{frame}

\begin{frame}[plain]
\frametitle{Ruby on Rails = $?$}
\begin{block}{Ruby}
    Język programowania
\end{block}
\begin{block}{Rails}
    Framework MVC
\end{block}
\begin{block}{Narzędzia}
    I kilka przydatnych bibliotek
\end{block}
\end{frame}

\mysection{Ruby}

\begin{frame}[fragile]\frametitle{Ruby}
    \begin{block}{Język programowania}
    \begin{itemize}
        \item skryptowy (interpretowany)
        \item zorientowany obiektowo
        \item dynamicznie typowany
    \end{itemize}
    \end{block}
\only<2->{\crossout}
\end{frame}

\mytitle{\centering \huge \color{orange} Übung macht den Meister \par}

\subsection{Jak zacząć?}
\begin{frame}[fragile]
\frametitle{Wypróbuj Ruby'ego bez instalacji}
    \myurl{tryruby.org}
    \myfullimage{tryruby.png}
\end{frame}
\begin{frame}[fragile]
\frametitle{Koduj albo giń!}
    \myurl{www.bloc.io/ruby-warrior}
    \myfullimage{rubywarrior.png}
\end{frame}

%%%% OBIEKTY %%%%%
\mytitle{Wszystko jest obiektem}

\begin{frame}[fragile]
\frametitle{Wszystko jest obiektem}
\begin{block}{}
\begin{lstlisting}[language=Ruby]
5.class  # => Fixnum
5 + 3    # => 8
5.+(3)   # => 8
\end{lstlisting}
\end{block}
\end{frame}

\begin{frame}[fragile]
\frametitle{Nic nie jest wieczne}
\begin{block}{}
\begin{lstlisting}[language=Ruby]
5.class  # => Fixnum
5 + 3    # => 8
5.+(3)   # => 8
\end{lstlisting}
\end{block}
\begin{block}{Psikus!}
\begin{lstlisting}[language=Ruby]
class Fixnum
  def +(x)
    self - x
  end
end

5 + 3    # => 2
\end{lstlisting}
\end{block}
\end{frame}

%%%% STALE %%%%%

\begin{frame}[fragile]
\frametitle{Nic nie jest wieczne}
\begin{block}{Stałe}
\begin{lstlisting}[language=Ruby]
CONSTANT = 5
\end{lstlisting}
\end{block}
\end{frame}

\begin{frame}[fragile]
\frametitle{Nic nie jest wieczne}
\begin{block}{Stałe}
\begin{lstlisting}[language=Ruby]
CONSTANT = 5
\end{lstlisting}
\end{block}
\begin{block}{Psikus!}
\begin{lstlisting}[language=Ruby]
CONSTANT = 5
CONSTANT = 0
# warning: already initialized constant CONSTANT
puts CONSTANT
# => 0
\end{lstlisting}
\end{block}
\end{frame}

\subsection{Zalety}
\mytitle{Czy Ruby ma jakieś zalety?}
\mytitle{Ma.}

\begin{frame}[fragile]
\frametitle{Elegancja}
\begin{block}{Pętla}
\begin{lstlisting}[language=Ruby]
5.times { puts "Hello!" }
Hello
Hello
Hello
Hello
Hello
\end{lstlisting}
\end{block}
\end{frame}

\begin{frame}[fragile]
\frametitle{Elegancja}
\begin{block}{Losowanie}
\begin{lstlisting}[language=Ruby]
gracze = ['Ela', 'Ola', 'Agnieszka', 'Marek', 'Wojtek', 'Marcin']
gracze.shuffle.each_slice(2).map do |g1, g2|
  "#{g1} i #{g2}"
end
\end{lstlisting}
\end{block}
\end{frame}

\begin{frame}[fragile]
\frametitle{Elegancja}
\begin{block}{Podnieś do kwadratu wszystkie liczby w tablicy}
\begin{lstlisting}[language=Ruby]
tablica = []
for i in [1,2,3,4]
  tablica << i ** 2
end
tablica   # => [1,4,9,16]
\end{lstlisting}
\end{block}
\only<2->{\crossout}
\end{frame}

\begin{frame}[fragile]
\frametitle{Elegancja}
\begin{block}{Podnieś do kwadratu wszystkie liczby w tablicy}
\begin{lstlisting}[language=Ruby]
[1,2,3,4].map { |i| i ** 2 }  # => [1,4,9,16]
\end{lstlisting}
\end{block}
\end{frame}

\begin{frame}[fragile]
\frametitle{Elegancja}
\begin{block}{Sortowanie według ostatniej litery imienia?}
\begin{lstlisting}[language=Ruby]
["Stefan", "Tomek", "Agata", "Maja"].sort_by do |imie|
  imie[-1]
end
# => ["Maja", "Agata", "Tomek", "Stefan"]
\end{lstlisting}
\end{block}
\end{frame}

\begin{frame}[fragile]
\frametitle{Elegancja}
\begin{block}{Podnieś do kwadratu wszystkie liczby w tablicy}
\begin{lstlisting}[language=Ruby]
[1,2,3,4].map { |i| i ** 2 }  # => [1,4,9,16]
\end{lstlisting}
\end{block}
\end{frame}
%%%% LICZBY %%%%

\begin{frame}[fragile]
\frametitle{Duże liczby?}
\begin{block}{No problem!}
\begin{lstlisting}[language=Ruby]
7 ** 200
  => 1046183829131435717501889961181681
365981918855017023365995014008403512576
742426225177438261490936405029306524825
254631417406318034368359118815075426733
9816534637456120001
\end{lstlisting}
\end{block}
\end{frame}




\begin{frame}[fragile]
\frametitle{bloki (closures)}
\end{frame}

\begin{frame}[fragile]
\frametitle{spaceship operator + sorting}
\end{frame}

\begin{frame}[fragile]
\frametitle{singleton classes}
\end{frame}

\begin{frame}[fragile]
\frametitle{CONSTANTS??}
\end{frame}

\begin{frame}[fragile]
\frametitle{Gems for everything}
 + Gemfile - so cool (bundle install)
\end{frame}

\begin{frame}[fragile]
\frametitle{Garbage Collection - Mark \& Sweep}
% http://patshaughnessy.net/2013/10/24/visualizing-garbage-collection-in-ruby-and-python
\end{frame}

\begin{frame}[fragile]
\frametitle{Wątki zielone}
\end{frame}
\subsection{Implementacje}
 - otwartość

\begin{frame}[fragile]
\frametitle{implementacje}
    * MRI
    - JRuby - różnice
\end{frame}


\subsection{Z czym się tego nie je}

\begin{frame}[fragile]
\frametitle{Z czym się tego nie je?}
\begin{itemize}
\item{Gry komputerowe}
    \begin{itemize}
        \only<2->{\item{Czyszczenie pamięci (mark \& sweep)}}
    \end{itemize}
        \only<3->{\item{Zadania wymagające wysokiej wydajności}}
    \begin{itemize}
      \only<4->{\item{kodu w Ruby'ego nie da się kompilować (w większości wersji)}}
      \only<5->{\item{można integrować z modułami C++}}
    \end{itemize}
        \only<6->{\item{Wielowątkowość}}
    \begin{itemize}
        \only<7->{\item{Zielone wątki, GIL}}
    \end{itemize}
        \only<8->{\item{Grafika, wykresy, analiza danych}}
    \begin{itemize}
        \only<9->{\item{Brak dobrych bibliotek}}
    \end{itemize}
\end{itemize}
\end{frame}

\section{Rails}
\mytitle{Oryginalna przyczyna popularności Ruby'ego?}
\mytitle{Technologie sieciowe}
\mytitle{Rails}

\subsection{O co chodzi?}
\begin{frame}[fragile]
\frametitle{Rails}
\begin{block}{Framework do szybkiego tworzenia aplikacji sieciowych}
\begin{itemize}
\item{Architektura MVC (Model - Widok - Kontroler)}
\item{Przydatne rozszerzenia podstawowych klas Ruby'ego (String, Time, \ldots)}
\item{Prosty serwer HTTP (Rack)}
\item{Integracja z wieloma bazami danych (SQL i NoSQL)}
\item{Convention over Configuration}
\item{Nacisk na TDD}
\end{itemize}
\end{block}
\end{frame}

\mytitle{Gdzie wykorzystano Ruby on Rails?}

\subsection{Zastosowanie}
\begin{frame}[fragile]
\frametitle{GitHub}
\myfullimage{github.png}
\end{frame}

\begin{frame}[fragile]
\frametitle{Twitter (początkowo)}
\myfullimage{twitter.png}
\end{frame}

\begin{frame}[fragile]
\frametitle{Hulu}
\myfullimage{hulu.png}
\end{frame}


\begin{frame}[fragile]
\frametitle{Co jeszcze dał nam Ruby?}

\begin{itemize}
\item{LessCSS / SASS}
\item{HAML}
\item{CoffeeScript}
\item{Homebrew}
\end{itemize}
\end{frame}

\section{Podsumowanie}

\mytitle{Co dalej?}

\begin{frame}[fragile]
\frametitle{Nauka}

WZORY CHEMICZNE PROBOWKI I TAKIE TAM****

\end{frame}

\begin{frame}[fragile]
\frametitle{Nauka}
\myurl{gorails.com}
\myfullimage{gorails.png}
\end{frame}

\begin{frame}[fragile]
\frametitle{Nauka}
\myurl{railscasts.com}
\myfullimage{railscasts.png}
\end{frame}

\begin{frame}[fragile]
\frametitle{Nauka}
\myurl{www.railstutorial.org}
\myfullimage{railstutorial.png}
\end{frame}

\begin{frame}[fragile]
\frametitle{Darmowy serwer}
\myurl{heroku.com}
\myfullimage{heroku.png}
\end{frame}

\begin{frame}[fragile]
\frametitle{Spotkania co miesiąc!}
\myurl{prug.pl}
\myfullimage{prug.png}
\end{frame}

\mytitle{Wykłady\par
\vspace{0.5cm}
\includegraphics[width=3cm]{piwo.png}}
}
\mytitle{Prelekcje\par
\vspace{0.5cm}
\includegraphics[width=3cm]{piwo.png}}
}
\mytitle{Spotkania z autorami różnych ciekawych bibliotek\par
\vspace{0.5cm}
\includegraphics[width=3cm]{piwo.png}}
}

\mytitle{Możliwość poznania poznańskich programistów Ruby on Rails}

\mytitle{A jeszcze lepiej się poznacie, jeśli zaaplikujecie do którejś z firm!****}

\begin{frame}[fragile]
\frametitle{Rekrutujemy!}
\myurl{netguru.co}
\myfullimage{rekrutacja.png}
\end{frame}

\begin{frame}[plain]
\begin{centering}
    \vspace{1em}\par
        \begin{beamercolorbox}[rounded=true,sep=4pt,center]{part title}
           \huge Dziękuję za uwagę!
        \end{beamercolorbox}
    \vspace{0.5cm}
    \myurl{github.com/marras/infomeet}
    \vspace{1.5cm}
    \includegraphics[width=7.5cm]{netguru_logo.jpg} \par
\end{centering}
\end{frame}
\end{document}