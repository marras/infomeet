%%%% OBIEKTY %%%%%
\mytitle{Wszystko jest obiektem}

\begin{frame}[fragile]
\frametitle{Wszystko jest obiektem}
\begin{block}{}
\begin{lstlisting}[language=Ruby]
5.class  # => Fixnum
5 + 3    # => 8
5.+(3)   # => 8
\end{lstlisting}
\end{block}
\end{frame}

\begin{frame}[fragile]
\frametitle{Nic nie jest wieczne}
\begin{block}{}
\begin{lstlisting}[language=Ruby]
5.class  # => Fixnum
5 + 3    # => 8
5.+(3)   # => 8
\end{lstlisting}
\end{block}
\begin{block}{Psikus!}
\begin{lstlisting}[language=Ruby]
class Fixnum
  def +(x)
    self - x
  end
end

5 + 3    # => 2
\end{lstlisting}
\end{block}
\end{frame}

%%%% STALE %%%%%

\begin{frame}[fragile]
\frametitle{Nic nie jest wieczne}
\begin{block}{Stałe}
\begin{lstlisting}[language=Ruby]
CONSTANT = 5
\end{lstlisting}
\end{block}
\end{frame}

\begin{frame}[fragile]
\frametitle{Nic nie jest wieczne}
\begin{block}{Stałe}
\begin{lstlisting}[language=Ruby]
CONSTANT = 5
\end{lstlisting}
\end{block}
\begin{block}{Psikus!}
\begin{lstlisting}[language=Ruby]
CONSTANT = 5
CONSTANT = 0
# warning: already initialized constant CONSTANT
puts CONSTANT
# => 0
\end{lstlisting}
\end{block}
\end{frame}

\subsection{Zalety}
\mytitle{Czy Ruby ma jakieś zalety?}
\mytitle{Ma.}

\begin{frame}[fragile]
\frametitle{Elegancja}
\begin{block}{Pętla}
\begin{lstlisting}[language=Ruby]
5.times { puts "Hello!" }
Hello
Hello
Hello
Hello
Hello
\end{lstlisting}
\end{block}
\end{frame}

\begin{frame}[fragile]
\frametitle{Elegancja}
\begin{block}{Losowanie}
\begin{lstlisting}[language=Ruby]
gracze = ['Ela', 'Ola', 'Agnieszka', 'Marek', 'Wojtek', 'Marcin']
gracze.shuffle.each_slice(2).map do |g1, g2|
  "#{g1} i #{g2}"
end
\end{lstlisting}
\end{block}
\end{frame}

\begin{frame}[fragile]
\frametitle{Elegancja}
\begin{block}{Podnieś do kwadratu wszystkie liczby w tablicy}
\begin{lstlisting}[language=Ruby]
tablica = []
for i in [1,2,3,4]
  tablica << i ** 2
end
tablica   # => [1,4,9,16]
\end{lstlisting}
\end{block}
\only<2->{\crossout}
\end{frame}

\begin{frame}[fragile]
\frametitle{Elegancja}
\begin{block}{Podnieś do kwadratu wszystkie liczby w tablicy}
\begin{lstlisting}[language=Ruby]
[1,2,3,4].map { |i| i ** 2 }  # => [1,4,9,16]
\end{lstlisting}
\end{block}
\end{frame}

\begin{frame}[fragile]
\frametitle{Elegancja}
\begin{block}{Sortowanie według ostatniej litery imienia?}
\begin{lstlisting}[language=Ruby]
["Stefan", "Tomek", "Agata", "Maja"].sort_by do |imie|
  imie[-1]
end
# => ["Maja", "Agata", "Tomek", "Stefan"]
\end{lstlisting}
\end{block}
\end{frame}

\begin{frame}[fragile]
\frametitle{Elegancja}
\begin{block}{Podnieś do kwadratu wszystkie liczby w tablicy}
\begin{lstlisting}[language=Ruby]
[1,2,3,4].map { |i| i ** 2 }  # => [1,4,9,16]
\end{lstlisting}
\end{block}
\end{frame}
%%%% LICZBY %%%%

\begin{frame}[fragile]
\frametitle{Duże liczby?}
\begin{block}{No problem!}
\begin{lstlisting}[language=Ruby]
7 ** 200
  => 1046183829131435717501889961181681
365981918855017023365995014008403512576
742426225177438261490936405029306524825
254631417406318034368359118815075426733
9816534637456120001
\end{lstlisting}
\end{block}
\end{frame}




\begin{frame}[fragile]
\frametitle{bloki (closures)}
\end{frame}

\begin{frame}[fragile]
\frametitle{spaceship operator + sorting}
\end{frame}

\begin{frame}[fragile]
\frametitle{singleton classes}
\end{frame}

\begin{frame}[fragile]
\frametitle{CONSTANTS??}
\end{frame}

\begin{frame}[fragile]
\frametitle{Gems for everything}
 + Gemfile - so cool (bundle install)
\end{frame}

\begin{frame}[fragile]
\frametitle{Garbage Collection - Mark \& Sweep}
% http://patshaughnessy.net/2013/10/24/visualizing-garbage-collection-in-ruby-and-python
\end{frame}

\begin{frame}[fragile]
\frametitle{Wątki zielone}
\end{frame}
\subsection{Implementacje}
 - otwartość

\begin{frame}[fragile]
\frametitle{implementacje}
    * MRI
    - JRuby - różnice
\end{frame}
