\mytitle{Dlaczego warto poznać Ruby'ego?}
%%%% OBIEKTY %%%%%
\mytitle{Wszystko jest obiektem}

\begin{frame}[fragile]
\frametitle{Wszystko jest obiektem}
\begin{block}{}
\begin{lstlisting}[language=Ruby]
5.class  # => Fixnum
5 + 3    # => 8
5.+(3)   # => 8
\end{lstlisting}
\end{block}
\end{frame}

\mytitle{Nic nie jest wieczne}
\begin{frame}[fragile]
\frametitle{Nic nie jest wieczne}
\begin{block}{}
\begin{lstlisting}[language=Ruby]
5.class  # => Fixnum
5 + 3    # => 8
5.+(3)   # => 8
\end{lstlisting}
\end{block}
\begin{block}{Psikus!}
\begin{lstlisting}[language=Ruby]
class Fixnum
  def +(x)
    self - x
  end
end

5 + 3    # => 2
\end{lstlisting}
\end{block}
\end{frame}

%%%% STALE %%%%%

\mytitle{Język dynamiczny}
\mytitle{Bardzo dynamiczny}
\begin{frame}[fragile]
\frametitle{Nic nie jest wieczne}
\begin{block}{Stałe}
\begin{lstlisting}[language=Ruby]
CONSTANT = 5
\end{lstlisting}
\end{block}
\end{frame}

\begin{frame}[fragile]
\frametitle{Nic nie jest wieczne}
\begin{block}{Stałe}
\begin{lstlisting}[language=Ruby]
CONSTANT = 5
\end{lstlisting}
\end{block}
\begin{block}{Psikus!}
\begin{lstlisting}[language=Ruby]
CONSTANT = 5
CONSTANT = 0
# warning: already initialized constant CONSTANT
puts CONSTANT
# => 0
\end{lstlisting}
\end{block}
\end{frame}

\mytitle{With great power comes great responsibility.}

\subsection{Zalety}
\mytitle{Czy Ruby ma jakieś zalety?}
\mytitle{Ma.}

%%%% ZMIENNE %%%

\begin{frame}[fragile]
\frametitle{Elegancja}
\begin{block}{Pętla}
\begin{lstlisting}[language=C++]
for (int i=0; i<5; i++) {
  printf("Hello!\n");
}
\end{lstlisting}
\end{block}
\only<2->{\crossout}
\end{frame}

\mytitle{Nie twórzmy niepotrzebnych zmiennych}

\begin{frame}[fragile]
\frametitle{Elegancja}
\begin{block}{Pętla - zmienna niepotrzebna}
\begin{lstlisting}[language=Ruby]
5.times { puts "Hello!" }
\end{lstlisting}
\end{block}
\end{frame}
\mytitle{Zmienna grzecznie czeka na drugim planie}
\begin{frame}[fragile]
\frametitle{Elegancja}
\begin{block}{Ruby nie narzuca ograniczeń}
\begin{lstlisting}[language=Ruby]
5.times { |i| puts i }

0
1
2
3
4
\end{lstlisting}
\end{block}
\end{frame}
\begin{frame}[fragile]
\frametitle{Elegancja}
\begin{block}{Ruby brzmi jak prawdziwy język}
\begin{lstlisting}[language=Ruby,basicstyle=\tiny\ttfamily]
exit unless "restaurant".include? "aura"

['toast', 'cheese', 'wine'].each { |food| print food.capitalize }

population = 12_000_000_000

(1..10).include? 5
\end{lstlisting}
\end{block}
\end{frame}

\mytitle{Jedna linijka}

\begin{frame}[fragile]
\frametitle{Elegancja}
\begin{block}{Podnieś do kwadratu wszystkie liczby w tablicy}
\begin{lstlisting}[language=Ruby]
tablica = []
for i in [1,2,3,4]
  tablica << i ** 2
end
tablica   # => [1,4,9,16]
\end{lstlisting}
\end{block}
\only<2->{\crossout}
\end{frame}
\begin{frame}[fragile]
\frametitle{Elegancja}
\begin{block}{Podnieś do kwadratu wszystkie liczby w tablicy}
\begin{lstlisting}[language=Ruby]
(1..4).map { |i| i ** 2 }  # => [1,4,9,16]
\end{lstlisting}
\end{block}
\end{frame}

\begin{frame}[fragile]
\frametitle{Elegancja}
\begin{block}{Sortowanie według ostatniej litery imienia?}
\begin{lstlisting}[language=Ruby,basicstyle=\tiny\ttfamily]
["Stefan", "Tomek", "Agata", "Maja"].sort_by { |imie| imie[-1] }

# => ["Maja", "Agata", "Tomek", "Stefan"]
\end{lstlisting}
\end{block}
\end{frame}
%%%%% TURNIEJ %%%%%%
\begin{frame}[plain]
\myfullimage{foosball.jpg}
\end{frame}

\begin{frame}[fragile]
\frametitle{Elegancja}
\begin{block}{Losowanie drużyn do turnieju}
\begin{lstlisting}[language=Ruby]
gracze = ["Ela", "Ola", "Agnieszka",
          "Marek", "Wojtek", "Marcin"]
\end{lstlisting}
\end{block}
\end{frame}
\begin{frame}[fragile]
\frametitle{Elegancja}
\begin{block}{Losowanie drużyn do turnieju}
\begin{lstlisting}[language=Ruby]
gracze = ["Ela", "Ola", "Agnieszka",
          "Marek", "Wojtek", "Marcin"]

gracze.shuffle.each_slice(2).to_a

# => [["Agnieszka", "Marek"],
#     ["Wojtek", "Ola"],
#     ["Marcin", "Ela"]]
\end{lstlisting}
\end{block}
\end{frame}

%%%% LICZBY %%%%

\begin{frame}[fragile]
\frametitle{Duże liczby?}
\begin{block}{No problem!}
\begin{lstlisting}[language=Ruby]
7 ** 200
  => 1046183829131435717501889961181681
365981918855017023365995014008403512576
742426225177438261490936405029306524825
254631417406318034368359118815075426733
9816534637456120001

7.class
# => Fixnum
(7 ** 200).class
# => Bignum
\end{lstlisting}
\end{block}
\end{frame}


%%%% METAPROGRAMMING %%%%%%
\mytitle{Meta-programowanie}

\begin{frame}[fragile]
\frametitle{Metaprogramowanie}
\begin{block}{Sztuczne metody}
\begin{lstlisting}[language=Ruby]
class Auto
  def wypisz_predkosc
    puts "predkosc = #{predkosc}"
  end

  def wypisz_moc
    puts "moc = #{moc}"
  end

  private

  def predkosc
    "200 km/h"
  end

  def moc
    120
  end
end
\end{lstlisting}
\end{block}
\only<2->{\crossout}
\end{frame}
\begin{frame}[plain]
    \centering \includegraphics[height=\paperheight]{dry_yoda.jpg} \par
\end{frame}
\mytitle{obiekt.send(nazwa\_metody)}
\begin{frame}[fragile]
\frametitle{Metaprogramowanie}
\begin{block}{Wywołanie wybranej metody}
\begin{lstlisting}[language=Ruby]
class Auto
  def wypisz(co)
    puts "#{co} = #{send(co)}"
  end

  private

  def predkosc
    "200 km/h"
  end

  def moc
    120
  end
end

a = Auto.new
a.wypisz('moc')
# moc = 120

\end{lstlisting}
\end{block}
\end{frame}

\begin{frame}[plain]
\begin{block}{}
\begin{centering}
\color{violet}Ruby
   
\only<1->{\color{black}potrafi wywołać metodę}
   
\only<2->{która}
   
\only<3->{\huge \color{red} NIE ISTNIEJE}

\end{centering}
\end{block}
\end{frame}

\begin{frame}[fragile]
\frametitle{Metaprogramowanie}
\begin{block}{Wywołanie nieistniejącej metody}
\begin{lstlisting}[language=Ruby]
class A
  def hello
    puts "Hello, world!"
  end
end

a = A.new
a.hello
# "Hello, world!"
a.inna_metoda
# NoMethodError: undefined method `inna_metoda' for a:A
\end{lstlisting}
\end{block}
\end{frame}

\mytitle{Obiekty są samoświadome}
\begin{frame}[plain]
  TU OBARZEK HAL'A
\end{frame}

\begin{frame}[fragile]
\frametitle{Metaprogramowanie}
\begin{block}{Pomocny program}
\begin{lstlisting}[language=Ruby,basicstyle=\tiny\ttfamily]
class A
  def hello
    puts "Hello, world!"
  end

  private

  def method_missing(name, *args)
    puts "Czy miales na mysli: #{self.methods.join(', ')}"
  end
end

a = A.new
a.inna_metoda
# Czy miales na mysli: hello, nil?, ===, =~, !~, eql?, hash, <=>, class, singleton_class, clone, dup, taint, tainted?, untaint, untrust, untrusted?, trust, freeze, frozen?, to_s, inspect, methods, singleton_methods, protected_methods, private_methods, public_methods, instance_variables, instance_variable_get, instance_variable_set, instance_variable_defined?, remove_instance_variable, instance_of?, kind_of?, is_a?, tap, send, public_send, respond_to?, extend, display, method, public_method, singleton_method, define_singleton_method, object_id, to_enum, enum_for, ==, equal?, !, !=, instance_eval, instance_exec, __send__, __id__
\end{lstlisting}
\end{block}
\end{frame}
\begin{frame}[fragile]
\frametitle{Metaprogramowanie}
\begin{block}{Jeszcze bardziej pomocny}
\begin{lstlisting}[language=Ruby,basicstyle=\tiny\ttfamily]
class A
  def hello
    puts "Hello, world!"
  end

  private

  def method_missing(name, *args)
    metody = self.methods - Object.methods
    puts "Czy miales na mysli: #{metody.join(', ')}"
  end
end

a = A.new
a.inna_metoda
# Czy miales na mysli: hello
\end{lstlisting}
\end{block}
\end{frame}


\begin{frame}[fragile]
\frametitle{bloki (closures)}
\end{frame}

\begin{frame}[fragile]
\frametitle{singleton classes}
\end{frame}

\begin{frame}[fragile]
\frametitle{Gems for everything}
 + Gemfile - so cool (bundle install)
\end{frame}

\begin{frame}[fragile]
\frametitle{Garbage Collection - Mark \& Sweep}
% http://patshaughnessy.net/2013/10/24/visualizing-garbage-collection-in-ruby-and-python
\end{frame}

\begin{frame}[fragile]
\frametitle{Wątki zielone}
\end{frame}

\begin{frame}[fragile]
\frametitle{implementacje}
    * MRI
    - JRuby - różnice
\end{frame}
